\apendice{Plan de Proyecto Software}

\section{Introducción}\label{introduccion-plan}
En este capitulo se detalla la planificación del proyecto. Como gestor de tareas se comenzó utilizando \emph{Trello+Github} pero más tarde se pasó a utilizar \emph{Zenhub,} extensión de Google Chrome que permiten integrar los \emph{boards} dentro del mismo repositorio de código alojado en \emph{github}. Ya se han dado más detalles en la memoria del proyecto.

Se ha utilizado metodologías ágiles para el desarrollo del proyecto y de este modo, se ha realizado un desarrollo dividido en iteraciones. Terminada una iteración empezaba la siguiente y se agregaban a las tareas planeadas las que no habían sido completado de la iteracción precedente. Las iteraciones del proyecto estaban pensadas para durar una diez días aproximadamente. No obstante, hay alguna excepción en la que la iteración duró más tiempo. También existe alguna demora entre algún sprint debido a que tenía demasiada carga de trabajo de las asignaturas, trabajaba o estaba de viaje. 

La fase de planificación se puede dividir a su vez en:

\begin{itemize}
\tightlist
\item
  Planificación temporal.
\item
  Estudio de viabilidad.
\end{itemize} 

La primera parte me centro en la programación y desarrollo de la aplicación.Es decir elaboro un programa de tiempos con una serie de tareas a seguir para cumplimentar el proyecto.

La segunda parte se centra en el estudio de viabilidad. De la misma manera desde la segunda semana de marzo vengo realizando un plan de empresa con el programa Yuzz por lo que ello me va a facilitar el estudio de viabilidad de mi proyecto. Se desarrollará tanto la viabilidad legal como también la económica. 


\section{Planificación temporal}\label{planificacion-temporal}
Desde inicio del proyecto se planteó utilizar una metodología ágil como
\emph{Scrum} para la gestión del proyecto. Aunque no se ha seguido al 100\% la
metodología al tratarse de un proyecto para la Universidad, sí que se ha aplicado
en líneas generales una filosofía ágil y metódica.

A continuación se describen los diferentes \emph{sprints} que se han
realizado. Dentro de \emph{github} cada \emph{milestone} recibe el número del sprint asignado y dentro de cada uno de ellos existen diferentes tareas que describiré a continuación. A cada tarea le acompaña un número a la derecha el cuál es denominado \emph{story point}, de alguna manera sirve para realizar una estimación de lo que te va llevar completar esa determinada tarea. En mi caso concreto el 1 resulta ser el más bajo lo cuál indicaría que no más de dos o tres horas con cada tarea y el 13 el más alto lo cuál significa días de trabajo.


\subsection{Sprint 0: 18/02/2017 - 28/02/2017}\label{sprint0}

Tareas principales:

\begin{itemize}
	\item Terminar formación.
	\item Aprender \emph{Sonarqube}.
	\item Inicio Back-End.
    \item Inicio Decidir base de datos a emplear.
\end{itemize}

\imagen{sprint0}{Detalle sprint 0}

La primera vez que hice uso de \emph{ZenHub} ya llevaba algún tiempo formándome, de ahí el primer \emph{Issue} del \underline{Sprint 0}. Esta primera toma de contacto fue para comenzar a desarrollar el Back-End de la aplicación, además se consultaron varias fuentes para decidir que tipo de base de datos emplear.

\subsection{Sprint 1: 18/02/2017 - 28/02/2017}\label{sprint1}

Tareas principales:

\begin{itemize}
	\item Correción de errores en Back-End.
	\item Desarrollo Back-end.
	\item Errores en base de datos.
\end{itemize}

\imagen{sprint1}{Detalle sprint 1}

El  \underline{Sprint 1} sirvió para continuar con el Back-end de la aplicación sin duda fue un de las partes más complicadas al pelearme con bases de datos con conceptos nuevos por lo que tuve numerosos bugs a la hora de guardar los usuarios en la base de datos.

\subsection{Sprint 2: 15/03/2017 - 31/03/2017}\label{sprint2}

Tareas principales:

\begin{itemize}
	\item Corrección de errores en Back-End.
	\item Inicio Front-End.
	\item Bug en base de datos.
	\item Comienzo a leer sobre la documentación.
\end{itemize}

\imagen{sprint2}{Detalle sprint 2}

El \underline{Sprint 2} fue el momento donde una vez tenía un back-end sólido debía trasladarlo a la parte del usuario por lo que comencé a realizar la parte del front-end.

\subsection{Sprint 3: 07/04/2017 - 15/04/2017}\label{sprint3}

Tareas principales:

\begin{itemize}
	\item Subir documentación.
	\item Elección del calendario.
	\item Corrección en componentes.
\end{itemize}

\imagen{sprint3}{Detalle sprint 3}

El \underline{Sprint 3} se centra en la parte del calendario sobre todo, además de algo de documentación y corregir los errores que he arrastrado del back-end.

\subsection{Sprint 4: 16/04/2017 - 22/04/2017}\label{sprint4}

Tareas principales:

\begin{itemize}
	\item Actualizar a Angular CLI.
	\item Heroku y MLab
	\item Documentación
\end{itemize}

\imagen{sprint4}{Detalle sprint 4}

El \underline{Sprint 4} es más corto dado que requiere un menor tiempo en realizar las tareas.

\subsection{Sprint 5: 30/04/2017 - 07/05/2017}\label{sprint5}

Tareas principales:

\begin{itemize}
	\item Angular CLI.
	\item Bugs
\end{itemize}

\imagen{sprint5}{Detalle sprint 5}

El \underline{Sprint 5} fue complicado debido a que cambiar a Angular CLI resulta más sencillo a la hora de desplegar en servidor pero hay que saber como funciona realmente los proyectos en Angular CLI

\subsection{Sprint 6: 09/05/2017 - 16/05/2017}\label{sprint6}

Tareas principales:

\begin{itemize}
	\item Tarea1
	\item Tarea2
\end{itemize}

\imagen{sprint6}{Detalle sprint 6}

El \underline{Sprint 6} 

\subsection{Sprint 7: 24/05/2017 - 31/05/2017}\label{sprint7}

Tareas principales:

\begin{itemize}
	\item Tarea1
	\item Tarea2
\end{itemize}

\imagen{sprint7}{Detalle sprint 7}

El \underline{Sprint 7} 

\subsection{Sprint 8: 01/06/2017 - 10/06/2017}\label{sprint8}

Tareas principales:

\begin{itemize}
	\item Tarea1
	\item Tarea2
\end{itemize}

%\imagen{sprint8}{Detalle sprint 8}

El \underline{Sprint 8} 

\subsection{Sprint 9: 10/06/2017 - 20/06/2017}\label{sprint9}

Tareas principales:

\begin{itemize}
	\item Tarea1
	\item Tarea2
\end{itemize}

%\imagen{sprint9}{Detalle sprint 9}

El \underline{Sprint 9} 

\subsection{Sprint 10: 20/06/2017 - 30/06/2017}\label{sprint10}

Tareas principales:

\begin{itemize}
	\item Tarea1
	\item Tarea2
\end{itemize}

%\imagen{sprint10}{Detalle sprint 10}

El \underline{Sprint 10} 


\section{Estudio de viabilidad}\label{estudio-viabilidad}
En esta sección se lleva a cabo un estudio para comprobar la viabilidad del proyecto realizado. Paralelamente al desarrollo de la aplicación, como ya se ha nombrado en la memoria anteriormente, el proyecto formó parte del programa YUZZ para jóvenes emprendedores en el que durante cinco meses realicé un plan de empresa completo. Se detalla por tanto en un documento que adjuntaré al proyecto un estudio de viabilidad exhaustivo  y muy completo en el que se incluyen entre otras cosas: plan de marketing, plan de financiación, estudio de viabilidad o plan de puesta en marcha del negocio a cinco años vista. 

Por lo tanto en esta sección voy a realizar un resumen del documento descrito en el párrafo anterior en el que como conclusión definitiva tendremos un boceto de lo que supondría transformar un proyecto fin de carrera y que pase a formar parte del mercado. Así mismo voy a intentar adaptarlo a las condiciones que se exigen en el proyecto dado que el plan de empresa completo es un estudio de viabilidad completo de aquí a cinco años por lo que resulta ser más extenso y detallado. Se intentará por tanto aquí realizar una estimación.

\subsection{Viabilidad económica}\label{viabilidad-economica}

La viabilidad económica es la parte donde lograremos detectar si el proyecto es o no rentable económicamente hablando.

\subsubsection{Análisis de costes}\label{costes}
Económica
\begin{description}
	\item[Coste de personal] Se considerará que el proyecto ha sido desarrollado en un periodo de cinco meses.  Considerando que se ha trabajado unas 6 horas a día cada semana, y que el programador, que en este caso es una sola persona, ha percibido un sueldo de 13 e/hora, el coste del personal por lo tanto se resume en la siguiente tabla:
	
\begin{table}[htbp]
\begin{center}
\begin{tabular}{|l|l|}
\hline
 & Total \\
\hline \hline
13 \euro /hora * 6 horas/día &   78 \euro /día \\ \hline
78 \euro /día * 5 dias/semana &   390 \euro /semana \\ \hline
390 \euro /semana * 4 semanas/mes &   1560 \euro /mes \\ \hline
\textbf{Coste total salario} &   7800 \euro /5 meses \\ \hline
\end{tabular}
\caption{Tabla salarios.}
\label{tabla:salarios}
\end{center}
\end{table}
	
	\item[Coste de seguridad social] más información sobre el segundo item.
	
	\item[Coste de software] más información sobre el segundo item.
	
	\item[Coste de Hardware] más información sobre el segundo item.

\end{description}

\subsection{Viabilidad legal}\label{viabilidad-legal}
La viabilidad legal se centra principalmente en el estudio de las licencias software utilizadas y en la licencia que se le va a ser asignada a las diferentes aplicaciones desarrolladas.




