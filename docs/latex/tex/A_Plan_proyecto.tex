\apendice{Plan de Proyecto Software}

\section{Introducción}\label{introduccion-plan}
En este capitulo se detalla la planificación del proyecto. Como gestor de tareas se comenzó utilizando Trello+github pero más tarde se pasó a utilizar Zenhub, extensión de Chrome que permiten integrar los boards dentro de github. Una opción sin duda muchísimo mas cómoda. 
Se ha utilizado metodologías agiles para el desarrollo del proyecto y de este modo, se ha realizado un desarrollo dividido en iteraciones. Terminada una iteración empezaba la siguiente y se agregaban a las tareas planeadas las que no habían sido completado de la iteración precedente. Las iteraciones del proyecto estaban pensadas para durar una semana. No obstante, hay alguna excepción en la que la iteración duró dos semanas.

La fase de planificación se puede dividir a su vez en:

\begin{itemize}
\tightlist
\item
  Planificación temporal.
\item
  Estudio de viabilidad. (Plan de empresa) 
\end{itemize} 

La primera parte me centro en la programación y desarrollo de la aplicación. Utilicé el gestor git durante toda la programación así como la extensión Zenhub. Es decir elaboro un programa de tiempos (todo ello disponible online en github) con una serie de tareas a seguir para cumplimentar el proyecto.

La segunda parte se centra en el estudio de viabilidad. De la misma manera desde la segunda semana de marzo vengo realizando un plan de empresa con el programa Yuzz por lo que ello me va a facilitar el estudio de viabilidad de mi proyecto. Se desarrollará tanto la viabilidad legal como la económica. 


\section{Planificación temporal}\label{planificacion-temporal}
Al inicio del proyecto se planteó utilizar una metodología ágil como
Scrum para la gestión del proyecto. Aunque no se ha seguido al 100\% la
metodología al tratarse de un proyecto para la Universidad (no éramos un equipo de
4 a 8 personas, no hubo reuniones diarias, etc.), sí que se ha aplicado
en líneas generales una filosofía ágil.

A continuación se describen los diferentes \emph{sprints} que se han
realizado.


\subsection{Sprint 0 (09/01/17 -
15/01/17)}\label{sprint-0}

\subsection{Sprint 1}

\subsection{Sprint 2}
\subsection{Sprint 3}
\subsection{Sprint 4}
\subsection{Sprint 5}


\section{Estudio de viabilidad}\label{estudio-viabilidad}
Perfectamente detallado en el informe realizado para el YUZZ: plan de empresa de 102 páginas, no ya solo con la viabilidad de la herramienta sino con plan económico de aquí a cinco años vistas 


\subsection{Viabilidad económica}\label{viabilidad-economica}
Economica

\subsection{Viabilidad legal}\label{viabilidad-legal}
Legal


