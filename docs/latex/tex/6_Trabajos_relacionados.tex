\capitulo{6}{Trabajos relacionados}

En esta sección se pretende realizar un acercamiento a los posibles competidores que la empresa podría tener en un potencial mercado, en caso de que llegara a ponerse a la venta. En el plan de empresa realizado, del cual se dan más detalles en los anexos, se ha realizado un estudio de la competencia más exhaustivo. Aquí nombraremos las principales competencias y sus principales ventajas o desventajas respecto a este proyecto.

\section{Trabajos previos}
Indispensable hacer una mención a un trabajo fin de grado previo realizado en este mismo grado y universidad. Aunque no tiene nada que ver el desarrollo la idea y el tutor fueron el mismo luego era necesario nombrarlo.

\section{Competidores}

\begin{table}[!hbt]
\begin{center}
\begin{tabular}{|l|l|}
\hline
Nombre & \emph{Tipo} \\
\hline
\hyperlink{https://play.google.com/store/apps/details?id=ciesdesign.SaTurnos&hl=es}{Saturnos Pro} & Aplicación Android para smartphones\\
\hline
\hyperlink{https://itunes.apple.com/es/app/cuadraturnos-free-calendario-de-turnos-de-trabajo/id1054129506?mt=8/}{CuadraTurnos Free} & Aplicación Android e IOS para smartphones\\
\hline
\hyperlink{https://myshyft.com/}{Shyft}   & Aplicación Android e IOS para smartphones\\
\hline
\end{tabular}
\caption{Listado de posibles competidores}
\end{center}
\end{table}


En la tabla anterior  se describen tres aplicaciones. Se ha considerado incluir estas tres  por diferentes razones que se expondrán a continuación.

\begin{description}
	\item[\emph{Saturnos pro}] La primera de ellas fue la que más nombro la gente durante las encuestas realizadas para el plan de empresa ya nombrado. Si hay una aplicación usada en este campo en España sin duda es ésta, si bien es cierto que tiene algunas limitaciones como por ejemplo que el único medio para usarla sea un dispositivo móvil con Android. La principal desventaja quizás sea el precio pero también era la más conocida por lo que eso significa que sea usa aunque no masivamente.
	\item[\emph{Cuadraturnos Free}] Fue otra de las aplicaciones nombradas durante la encuesta aunque con mucha menor insistencia, tan solo tres o cuatro personas la conocían. Está disponible en las principales plataformas móviles y resulta fácil de usar pero la interfaz resulta ser un poco antigua .
	\item[\emph{Shyft}] Sin duda es la aplicación más reciente, sólo esta disponible en Estados Unidos pero es el modelo a fijarme como herramienta. La he probado y usado y quizás no resulta tan intuitiva como parece ser.
\end{description}

Nota: No

\section{Otros trabajos}
Aunque no directamente relaciones si he encontrado algún trabajo o proyecto que intentaban desarrollar algoritmos para suplir problemas de asignación de turnos. Me a parecido interesante e útil incluirlos en esta sección.
     \begin{itemize}
         \item http://www.laccei.org/LACCEI2013-Cancun/RefereedPapers/RP090.pdf
      %   \item http://www.unipamplona.edu.co/unipamplona/portalIG/home_40/recursos/03_v13_18/revista_16/27102011/13.pdf
    \end{itemize}
