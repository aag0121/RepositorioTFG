\apendice{Especificación de diseño}
\section{Introducción}\label{introducciondiseno}
Cada parte de la aplicación tiene características interesante de programación que merece la pena reflejar aquí: comprender qué datos se han usado, cómo se han pensado y el tipo de los mismos resulta fundamental para entender el comportamiento global de la aplicación.

\section{Diseño de datos}\label{datos}
Vamos a analizar los datos así como las técnicas o librerías más importantes de las que se han hecho uso para usar, enviar y recoger todos esos datos. Debemos recordar que se trata de una tecnología basada en componentes, todos los componentes en Angular tienen una estructura interna exactamente igual:

\imagen{component}{\footnotesize{Componente Angular . Fuente:  \url{http://blog.enriqueoriol.com/}.}}\label{component}

En la  imagen \ref{component} podemos ver que se define un \textbf{selector} que es la parte para introducir ese determinado componente en la plantilla, un \textbf{templateUrl} que define la propia plantilla html asociada a ese componente, \textbf{stylesUrls} que define la hoja de estilos asociada a esa plantilla y \textbf{module.id} que en nuestro caso no tiene demasiada importancia.  Los únicos estrictamente obligatorios son los dos primeros.

Si nos fijamos en las buenas prácticas de programación de Angular dictan que después del nombre de un componente viene la palabra \emph{component}, pasa lo mismo con, por ejemplo, los servicios cuya palabra en este caso será \emph{services}.

En la siguiente figura \ref{comphtml} podemos ver la relación entre un componente y su propia plantilla.

\imagen{comphtml}{\footnotesize{Relación plantilla-componente. Fuente:  \url{http://academia-binaria.com/}.}}\label{comphtml}


Vamos a analizar cada una de las partes esenciales de la aplicación:


%Analisis
\begin{itemize}

	\item \emph{/app/directives} : las directivas  en este caso se utilizan para lanzar alertas, concretamente en el registro o  de la aplicación. 
	
	\item \emph{/app/guards} : es la parte donde realmente se comprueba que un usuario este logeado en la web \cite{guards}. Me basé en un tutorial \cite{logintut} de Jason Watmore, programador australiano. 
	
	\item \emph{/app/models} : junto con \emph{services} una de las partes fundamentales de la aplicación es dónde se describen los datos donde se declaran los atributos que va a tener cada entidad que vamos a usar. En mi caso: \underline{usuarios} y \underline{eventos}.  
	Se ha usado aqui un DTO \cite{dto} para el caso de los eventos que van a ser intercambiados, es decir, ese turno que un usuario va a intercambiar con otro. Es habitual esta práctica de programación en java pero también empieza a serlo en Angular \cite{dto2}.
	
	A continuación vamos a ver cada una de las clases que componen y en las que se basa la aplicación
	
\begin{table}[H]
\centering
\begin{tabular}{|l|l|}
\hline
Tipo de dato & Parámetros  \\
\hline \hline
\textbf{Usuario} [user] & id: string \\ 
    & email: string \\
    & password: string \\
    & cname: string \\
    & firstName: string \\
    & lastName: string \\
    & shift: string \\
    & username: string \\ \hline
\end{tabular}
\caption{Clase Usuario}
\end{table}



\begin{table}[H]
\centering
\begin{tabular}{|l|l|}
\hline
Tipo  & Parámetros  \\
\hline \hline
\textbf{Evento} [event.dto] & id: string \\ 
    & start: Date \\
    & end: Date \\
    & title: string \\
  & primary\_color: string \\
  & secondary\_color: string \\
  & draggable: boolean \\
  & resizable\_beforeStart: boolean \\
  & resizable\_afterEnd: boolean \\
  & company: string \\
  & type: string \\
  & username: string \\
  & status: string \\ 
  & sender: string\\
  & turn\_in\_day: string \\ \hline
\end{tabular}
\caption{Clase Evento.}
\end{table}







\begin{table}[H]
\centering
\begin{tabular}{|l|l|}
\hline
Tipo  & Parámetros  \\
\hline \hline
\textbf{Evento del usuario} [my-event] & id: string \\ 
  & id: string \\ 
 & type: string \\ 
  & username: string \\ 
 & status: string \\  
 & sender: string \\ 
  & turn\_in\_day: string\\ \hline
\end{tabular}
\caption{Clase Evento-usuario.}
\end{table}





\begin{table}[H]
\centering
\begin{tabular}{|l|l|}
\hline
Tipo  & Parámetros  \\
\hline \hline
\textbf{Evento a intercambiar} [interchange.dto] & id: string \\ 
  & requestor: string \\ 
  & acknowledger: string \\ 
  & requestor\_event\_id: string \\ 
  & acknowledger\_event\_id: string \\ 
  & status: string;  \\ \hline
\end{tabular}
\caption{Clase Intercambio.}
\label{tabla:sencilla2}
\end{table}


\imagen{relacional}{\footnotesize{Diagrama relacional. Fuente: Elaboración propia.}}
	
	
	\item \emph{/app/services} : los servicios de la parte cliente, las alertas nombradas anteriormente o la autenticación tienen su propio servicio. Es interesante el conocer aquí el concepto de \emph{Observable} \cite{ob1} en Angular, a la hora de las comunicaciones http es algo muy a tener en cuenta. La librería \textbf{@angular/http} trae por defecto el servicio http que es el cliente usado para enviar y recibir datos \cite{ob2}, pero sin duda la librería más potente actualmente sobre el tema es \href{http://reactivex.io/rxjs/}{RxJS}. Recomiendo encarecidamente el artículo: Javascript Reactivo y funcional \cite{ob3}, que aborda estos temas con gran entereza.
	
	\item \emph{/app/pages} : comprende las distintas páginas \emph{fisicas} de la web.
	
	\item \emph{/app/repository} : es la conexión de datos entre la parte del cliente y la parte servidor.
	
	\imagen{schemametadato}{\footnotesize{Esquema Angular por dentro. Fuente:  \url{www.angular.io}.}}
	
	\item \emph{/app/utils} : es una carpeta extra que pretende emplear el patrón \emph{Singleton} definiendo datos inamovibles durante toda la vida de la aplicación. Se utiliza el nombre utils como buena práctica de programación ya que se trata de datos útiles que no van a cambiar nunca. De esta manera si queremos añadir datos nuevos en un futuro tan solo hay que añadirlos dentro de este componente. 
	
	\begin{table}[H]
\begin{center}
\begin{tabular}{|l|l|}
\hline
Definición & Datos  \\
\hline \hline
\textbf{Tipos de turno} [shifts] &  24 horas \\ \hline
 &  12 horas \\ \hline
 &  8 horas\\ \hline
 &  6 horas\\ \hline
 \hline \hline
  \textbf{Tipos de eventos}  [types] & Intocable \\ \hline
 &  Asignado conforme \\ \hline
 &  Asignado eliminar \\ \hline
 &  Día libre\\ \hline
 \hline \hline
  \textbf{Colores asociados a los eventos} [colors] & \textcolor{red}{Rojo} \small{(Intocable)} \\ \hline
 &  \textcolor{blue}{Azul} \small{(Asignado conforme)} \\ \hline
 &  \textcolor{yellow}{Amarillo} \small{(Asignado eliminar)} \\ \hline
 &  \textcolor{green}{Verde} \small{(Día libre)} \\ \hline
 \hline \hline
  \textbf{Turnos /día} [turnInDay] & 1 turno por día \\ \hline
 & 2 turnos por día  \\ \hline
 & 3 turnos por día  \\ \hline
 & 4 turnos por día  \\ \hline
 \hline \hline
   \textbf{Estados de un evento} [eventStatus] & Normal \\ \hline
 & Pendiente  \\ \hline
 & Aceptado   \\ \hline
 & Rechazado  \\ \hline
  & Requerido  \\ \hline
\end{tabular}
\caption{Datos.}
\end{center}
\end{table}

En términos de programación todos son arrays excepto los turnos por día que se declaran como un mapa cuya \underline{clave} es el argumento uno, dos, tres y cuatro del array \emph{shifts} que corresponde al tipo de turno, y el \underline{valor} son los turnos en los que se subdivide un día: jornada completa, dos turnos, tres turnos o cuatro turnos.

\item \emph{Calendario} : el calendario es la parte fundamental de la aplicación, se ha empleado un componente desarrollado por \href{https://mattlewis.me/}{Matt Lewis}. Un desarrollador británico que tiene numerosas aportaciones al mundo \emph{open source}. El calendario está en desarrollo actualmente como se puede ver en su página de \href{https://github.com/mattlewis92/angular-calendar}{github}. 
\begin{itemize}
\item \url{https://github.com/mattlewis92/angular-calendar}
\end{itemize}



\item Otros detalles a tener en cuenta sobre el cliente.
\begin{itemize}
	\item \emph{ngAfterViewInit} \cite{afterview} : para intentar entederlo diremos que es algo que es necesario que se inicialice antes de la vista principal. 
	\item \emph{PrimeNg}: se trata de una serie de componentes diseñados para Angular que resultan ser realmente útiles. Persiguen el diseño \emph{responsive} de cara al usuario y tienen una comunidad enorme detrás \cite{prime}. Por ejemplo yo he empleado Growl \cite{growl} que se utiliza para lanzar al usuarios los mensajes de confirmación de las diferentes peticiones.  
\end{itemize}

\item\emph{/server/config} : se define aquí la configuración de la base de datos, local o remota, necesaria para almacenar los datos de la aplicación. Este fichero junto con \textbf{config.js}, que se encuentra en la raíz de la carpeta \emph{server}, son los que hay que modificar para cambiar de una base de datos a otra. En un futuro no muy lejano se creará una variable de entorno para que sea por defecto capaz de detectar si es necesario una base de datos local o una remota. 

\item\emph{/server/controllers} : son los controladores de la parte del servidor. Emplean la librería  \emph{Router} \cite{router} nativa en express, para poder funcionar.

\imagen{conexiones}{\footnotesize{Envío-respuesta http . Fuente:  \url{http://expressjs.com/}.}}

\item\emph{/server/routes} : son las rutas necesarias para realizar correctamente las llamadas \emph{http}  y que devuelvan en cada momento los datos correspondientes. Una vez el usuario se haya autenticado de forma correcta, ya puede acceder a las diferentes zonas de la aplicación. Cada sección dentro esta aplicación necesitará al menos una llamada a la API, situadas en rutas privadas, es decir, que necesitan autorización. Sin esa autorización no se pueden realizar las llamadas \emph{http} necesarias. 

En el siguiente par de páginas vamos a ver de un vistazo rápido los métodos empleados para extraer los datos y los métodos \emph{http} que tiene asociado en cada ruta cada uno de ellos. 
\begin{landscape}
%\begin{adjustbox}{width=1.2\textwidth,center}
\begin{table}[H]
\centering
\begin{tabular}{|l|l|l|l|}
\hline
Tipo  & Métodos & Método http asociado \\
\hline \hline
\textbf{Eventos} & .create\_event()  & POST \\ \hline
    & .create\_events()  &  PUT \\ \hline
    & .list\_all\_events()  & GET \\ \hline
    & .list\_all\_events\_by\_company()  &  GET\\ \hline
    & .list\_all\_events\_by\_company\_by\_worker() & GET \\ \hline
    & .list\_all\_free\_events\_by\_company\_by\_day\_by\_shift\_except\_currentUser() & GET\\ \hline
    & .update\_event()  & UPDATE \\ \hline
    & .delete\_event() & DELETE \\ \hline
    \hline \hline
\end{tabular}
\caption{Métodos para \emph{event} y http asociados.}
\end{table}

\begin{table}[H]
\centering
\begin{tabular}{|l|l|l|l|}
\hline
Tipo  & Métodos & Método http asociado \\
\hline \hline
\textbf{Intercambio} & .accept\_shift()  & POST \\ \hline
    & .decline()  & POST  \\ \hline
    & .activate\_shift()  &  POST \\ \hline
    \hline \hline
\end{tabular}
\caption{Métodos para \emph{interchange} y http asociados.}
\end{table}


\begin{table}[H]
\centering
\begin{tabular}{|l|l|l|l|}
\hline
Tipo  & Métodos & Método http asociado \\
\hline \hline
\textbf{Usuarios} & .authenticate(email, password)  & POST \\ \hline
    & .register()  & POST \\ \hline
    & .getAll()  &  GET\\ \hline
    & .getCurrent()  &  GET\\ \hline
    & .update(id)  &  PUT \\ \hline
    & .delete(id)  & DELETE \\ \hline
    \hline \hline
\end{tabular}
\caption{Métodos para \emph{users} y http asociados.}
\end{table}

%\end{adjustbox}
  \end{landscape}

\item\emph{/server/services} : 	es la parte del servidor dónde realmente se producen todas las acciones, el intercambio entre usuarios esta implementado aquí. La idea es usar una \textbf{API}, es decir, \emph{un conjunto de subrutinas, funciones y procedimientos que ofrece cierta biblioteca para ser utilizado por otro software como una capa de abstracción} \cite{wikiapi}. En definitiva es una forma de describir la forma en la que los sitios web intercambian datos, en este caso el intercambio de datos es mediante \textbf{JSON}. Si detallamos un poco más veremos que lo que realmente esta implementado en el servidor es una \textbf{API REST}, es decir un tipo de arquitectura (\textbf{RE}presentational \textbf{S}tate \textbf{T}ransfer) de desarrollo web que se apoya en una API y en el estándar \emph{http}, por lo que todas las llamadas a la API se implementan como peticiones \emph{http}..
	
	 \imagen{servidor}{\footnotesize{Funcionamiento \emph{API}. Fuente:  \url{https://juanda.gitbooks.io/}.}}
	
	
\end{itemize}



\begin{landscape}
%\begin{adjustbox}{width=1.2\textwidth,center}

\imagenflotante{sequenceschema}{\footnotesize{Vista de la conexión entre el servidor y el cliente. Fuente:  Elaboración propia.}}
%\end{adjustbox}
  \end{landscape}

\section{Diseño arquitectónico}\label{darquitectura}
En cuanto a la arquitectura de Angular resulta, después de todo lo ya visto, entenderse de una manera fácil e intuitiva. La aplicación compuesta por tres capas siguiendo la aquitectura MVC (Modelo-Vista-Controlador):
\begin{itemize}

\item \underline{Modelo}: es la capa que se comunica directamente con la base de datos. Se encarga de todos los procedimientos necesarios para crear, eliminar, recuperar y actualizar información de la base de datos.

\item \underline{Vista}: capa de presentación, implementa los métodos relacionados con los componentes de la interfaz de la aplicación.

 \item \underline{Controlador}: se encarga de la comunicación entre la vista y el modelo. En la aplicación parte de la lógica se manejará dentro de la propia página utilizando JavaScript dentro de los controladores de Angular y otra parte se hará en el servidor de NodeJS.
 


\end{itemize}

\imagen{meanarq}{\footnotesize{Esquema Arquitectura. Fuente:  \url{http://www.cantabriatic.com/}.}}

Seguir la estructura MVC es muy útil, ya que proporciona una idea mental de dónde situar cada elemento evitando así a tarea de investigación cada vez que se quiera ampliar el contenido de la aplicación.







