\capitulo{1}{Introducción}


Hoy en día existen multitud de situaciones en las que se requiere un reparto de turnos o guardias entre un grupo de personas (por ejemplo personal sanitario en servicios hospitalarios, servicios de emergencias extrahospitalarias, bomberos, protección civil y muchos ejemplos más). En muchas de éstas ocasiones, el reparto de turnos planificado podría mejorarse fácilmente si se llevaran a cabo cambios bilaterales que a veces no se llegan a producir porque las partes implicadas simplemente desconocen que existe esa oportunidad de mejora, o porque intuyen que los costes de encontrar esa mejora y llegar a  hacerla posible son excesivamente elevados. Otras veces los cambios sí que pueden llegar a producirse pero tras un largo período de negociación.

Actualmente, tras un proceso de investigación inicial, he determinado que no existe una herramienta sencilla que presente una solución a este problema. Es más en servicios como los sanitarios el problema no esta ni si quiera digitalizado, se suele realizar simple y llanamente en papel. Por una parte esta no es una mala técnica pero la implantación de una herramienta digital accesible por cualquiera puede facilitar las cosas enormemente a nivel de tiempo, espacio y entendimiento. 

El mundo de las aplicaciones, tanto web como móvil, resulta extremadamente competitivo hoy en día y posicionarse en el mercado es realmente complejo. Por una parte está la tarea de adaptarse a las nuevas tecnologías que cambian constantemente, tanto web como móvil, pero también la aceptación que el posible cliente o usuario haga de tu futuro producto o servicio.

\section{Motivacion}\label{Motivacion}

Explicar por que he elegido este trabajo y no otro es también describir cuáles son mis objetivos personales de cara a un futuro cada vez más cercano.  Me gusta el desarrollo web y me apasiona la posibilidad de mejorar el mundo a través de la tecnología, si bien no tengo experiencia dentro del desarrollo web si que me gustaría encaminarme hacia ello en un futuro no muy lejano dónde resulta tener un lugar muy prometedor \citep{desarrolloweb1}.

\section{Estructura de la memoria}\label{estructura-de-la-memoria}

\begin{itemize}
\tightlist
\item
  \textbf{Introducción:} Aquí se explica la motivación que me ha llevado a escoger este trabajo y no otro, la descripción del problema propuesto y una breve introducción general a la solución que se ha pretendido dar. Así mismo también una estructura de toda la memoria 
\item
  \textbf{Objetivos del proyecto:} tanto a nivel académico, personal como los objetivos alcanzados con el proyecto. También se comentará lo que le espera en un futuro al proyecto, ya que me gustaría seguir con él.
\item
  \textbf{Conceptos teóricos:} breve explicación de los conceptos
  teóricos que he debido adquirir previamente para la realización del proyecto. Necesarios para comprender mejor cómo funciona el proyecto.
\item
  \textbf{Técnicas y herramientas:} software principal, metodologías empleadas durante el proyecto, \emph{frameworks} empleados y herramientas.
\item
  \textbf{Aspectos relevantes del desarrollo:} detalles sobre el la realización del proyecto.
\item
  \textbf{Trabajos relacionados:} otros aspectos del desarrollo del proyecto.
\item
  \textbf{Conclusiones y líneas de trabajo futuras:} conclusiones
  obtenidas tras la realización de este trabajo fin de grado y hacia donde va en el futuro.
\end{itemize}

Junto a la memoria se proporcionan los siguientes anexos:

\begin{itemize}
\tightlist
\item
  \textbf{Plan del proyecto software:} estudio de viabilidad y planificación del proyecto.
\item
  \textbf{Especificación de requisitos del software:} se describe la
  fase de análisis; los objetivos generales, el catálogo de requisitos
  del sistema y la especificación de requisitos necesarios para su correcto funcionamiento.
\item
  \textbf{Especificación de diseño:} se describe la fase de diseño de la aplicación tanto emph{front-end} como emph{back-end} y el paso de la interfaz web a la móvil.
\item
  \textbf{Manual del programador:} recoge los aspectos más relevantes
  relacionados con el código fuente y su correcto manejo por parte de un programador.
\item
  \textbf{Manual de usuario:} guía de usuario para el uso de la aplicación.
\end{itemize}

