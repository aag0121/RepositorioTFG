\capitulo{3}{Conceptos teóricos}

La parte del proyecto más ardua ha sido la de aprender a utilizar \emph{frameworks}  para crear web apps, para lo cual me he nutrido de diversos cursos online, tanto gratuitos como de pago. A continuación, en este apartado voy a relatar todas las tecnologías empleadas y a justificar el por que he empleado unas y no otras. Las técnicas y tecnologías como tal se verán en el siguiente apartado pero todas ellas presentan unos conceptos teóricos comunes que es necesario conocer. 
\section{Introducción}\label{teorico-introduccion}
El desarrollo de aplicaciones informáticas evoluciona continuamente para adaptarse a las tecnologías de la información y las comunicaciones (TIC). El auge de Internet y de la web ha influido notablemente en el desarrollo de software durante los últimos años. Hoy en día la interfaz de los sistemas de información se implementa utilizando tecnologías web que ofrecen numerosas ventajas tales como el uso de una interfaz uniforme y la mejora del mantenimiento del sistema. Sin embargo, la existencia de numerosos estándares y los intereses de los fabricantes de tecnologías web dificultan el desarrollo de este tipo de aplicaciones. 

En los principios de la informática las relaciones en entre los usuarios y los programas de los que hacían uso era muy diferente a lo que tenemos ahora. Ahora se potencia el diseño basado en usuario

\imagen{uxschema}{Esquema \emph{UI, UX IxD}, fuente \emph{kambrica.com}.}


\section{Concepto general del proyecto}\label{teorico-general}



\subsection{¿Angular JS o Angular 2?}\label{angular-js-o-angular-2}



\section{Análisis del sistema: MVC}\label{analisis-sistema-mvc}


\subsection{Arquitectura}\label{arquitectura}


\subsection{Algoritmo}\label{algoritmo}


\subsection{Detalles}\label{detalles}



\section{Secciones}

Las secciones se incluyen con el comando section.

\subsection{Subsecciones}

Además de secciones tenemos subsecciones.

\subsubsection{Subsubsecciones}

Y subsecciones. 


\section{Referencias}

Las referencias se incluyen en el texto usando cite \cite{wiki:latex}. Para citar webs, artículos o libros \cite{koza92}.


\section{Imágenes}

Se pueden incluir imágenes con los comandos standard de \LaTeX, pero esta plantilla dispone de comandos propios como por ejemplo el siguiente:

\imagen{escudoInfor}{Autómata para una expresión vacía}



\section{Listas de items}

Existen tres posibilidades:

\begin{itemize}
	\item primer item.
	\item segundo item.
\end{itemize}

\begin{enumerate}
	\item primer item.
	\item segundo item.
\end{enumerate}

\begin{description}
	\item[Primer item] más información sobre el primer item.
	\item[Segundo item] más información sobre el segundo item.
\end{description}
	
\begin{itemize}
\item 
\end{itemize}

\section{Tablas}

Igualmente se pueden usar los comandos específicos de \LaTeX o bien usar alguno de los comandos de la plantilla.

\tablaSmall{Herramientas y tecnologías utilizadas en cada parte del proyecto}{l c c c c}{herramientasportipodeuso}
{ \multicolumn{1}{l}{Herramientas} & App AngularJS & API REST & BD & Memoria \\}{ 
HTML5 & X & & &\\
CSS3 & X & & &\\
BOOTSTRAP & X & & &\\
JavaScript & X & & &\\
AngularJS & X & & &\\
Bower & X & & &\\
PHP & & X & &\\
Karma + Jasmine & X & & &\\
Slim framework & & X & &\\
Idiorm & & X & &\\
Composer & & X & &\\
JSON & X & X & &\\
PhpStorm & X & X & &\\
MySQL & & & X &\\
PhpMyAdmin & & & X &\\
Git + BitBucket & X & X & X & X\\
Mik\TeX{} & & & & X\\
\TeX{}Maker & & & & X\\
Astah & & & & X\\
Balsamiq Mockups & X & & &\\
VersionOne & X & X & X & X\\
} 
