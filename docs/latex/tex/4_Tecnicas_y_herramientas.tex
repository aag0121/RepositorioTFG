\capitulo{4}{Técnicas y herramientas}

Esta parte de la memoria tiene como objetivo presentar las técnicas metodológicas y las herramientas de desarrollo que se han utilizado para llevar a cabo el proyecto. Si se han estudiado diferentes alternativas de metodologías, herramientas, bibliotecas se puede hacer un resumen de los aspectos más destacados de cada alternativa, incluyendo comparativas entre las distintas opciones y una justificación de las elecciones realizadas. 
No se pretende que este apartado se convierta en un capítulo de un libro dedicado a cada una de las alternativas, sino comentar los aspectos más destacados de cada opción, con un repaso somero a los fundamentos esenciales y referencias bibliográficas para que el lector pueda ampliar su conocimiento sobre el tema.

He de decir que durante el desarrollo del proyecto tanto  las herramientas cómo las técnicas a utilizar han ido mutando debido a que yo mismo iba descubriendo nuevas técnicas o nuevas herramientas que mejoraban mi manera de hacer el código o me facilitaban mi manera de trabajar. Esto puede que haya sido uno de los mayores errores del proyecto.

Por hacer y citar un de mis mutaciones: al comienzo empecé con brackets como IDE, ya que era el entorno que utilizaba el profesor de los cursos que me recomendó el tutor. Después y atraído por el uso a diario de Eclipse como herramienta principal de trabajo decidí que podía ser una buena idea el emplearlo también para realizar la API, craso error dado que Eclipse no fue creado para ello por lo que me vi con que cada vez me encontraba con menos soporte para lenguajes, sin ir más lejos eclipse no presenta, actualmente, soporte para typescript, el lenguaje principal de desarrollo de angular2. Por fin di con el IDE perfecto para desarrollar entornos web: Atom.

\section{Herramientas}\label{Herramientas}
\subsection{Entorno de desarrollo: IDE}\label{entorno-desarrollo}
