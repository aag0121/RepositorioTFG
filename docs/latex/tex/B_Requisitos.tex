\apendice{Especificación de Requisitos}

\section{Introducción}\label{introduccion-requisitos}

Este anexo recoge la especificación de requisitos que define el comportamiento del sistema desarrollado. El objetivo principal de la Especificación de Requisitos del Sistema (\emph{ERS}) es servir como medio de comunicación entre clientes, usuarios, ingenieros de requisitos y desarrolladores.

La ERS es correcta si y sólo si todo requisito que figura en ella refleja alguna necesidad
real. La corrección de la ERS implica que el sistema implementado será el sistema
deseado. Se han seguido las recomendaciones del estándar IEEE 830 según la última versión del estándar IEEE 830. Las características deseables para una especificación de requisitos son:


\begin{enumerate}
	\item No ambigua.
	\item Completa
	\item  Verificable
	\item  Consistente
	\item  Clasificada
	\item  Modificable
	\item  Explorable
	\item  Utilizable durante las tareas de mantenimiento y uso
\end{enumerate}


\section{Objetivos generales}\label{objetivos-generales}
Los objetivos generales que se perseguían con el proyecto han sido:
\begin{itemize}
	\item Desarrollar una aplicación web para solventar el problema de la gestión de turnos.
	\item Formarse en tecnologías web.
	\item Aprender a realizar una aplicación móvil híbrida a través de diversas plataformas como por ejemplo PhoneGapp.
\end{itemize}


\section{Catálogo de requisitos}\label{catalogo-requisitos}
A continuación, se enumeran los requisitos específicos :

\subsection{Requisitos funcionales}\label{r-funcionales}

\section{Especificación de requisitos}\label{r-nfuncionales}



%Tabla grande
\begin{table}
\begin{tabular}{llr}  
\toprule
\begin{minipage}[b]{0.23\columnwidth}\raggedright\strut
\textbf{CU-02}\strut
\end{minipage} & \begin{minipage}[b]{0.71\columnwidth}\raggedright\strut
\textbf{Crear cuestionarios}\strut
\end{minipage}\tabularnewline
\cmidrule(r){1-2}
\textbf{Versión}       & 1.0           \\
\textbf{Autor}       & Adrián  Aguado    \\
\textbf{Requisitos asociados}       & RF-1.1, RF-1.2 \\ 
\textbf{Descripción} & Permite al usuario logearse\\
\textbf{Precondición} & LAs precondiciones \\
& deben estar disponibles        \\
\textbf{Secuencia} & 1. El usuario rellena el formulario \\
& para\\
& 2. El usuario \\
& 3. Si\\
& se        \\
\textbf{Postcondición} & Se añade el usuario \\
& a la base de datos.     \\
\textbf{Excepciones} &  Si no se introducen los campos obligatorios \\
& se muestra texto resaltando indicándolo
          \\
\textbf{Frecuencia} & Alta            \\
\textbf{Importancia} & Alta            \\
\textbf{Comentarios } & Esto es un comentario        \\
& sobre este requisito.     \\
\bottomrule
\end{tabular}
\caption{CU-02 Crear cuestionarios} \label{tab:sometab}
\end{table}