\apendice{Especificación de Requisitos}

\section{Introducción}\label{introduccion-requisitos}

Este anexo recoge la especificación de requisitos que define el comportamiento del sistema desarrollado. El objetivo principal de la Especificación de Requisitos del Sistema (\emph{ERS}) es servir como medio de comunicación entre clientes, usuarios, ingenieros de requisitos y desarrolladores.

La ERS es correcta si y sólo si todo requisito que figura en ella refleja alguna necesidad
real. La corrección de la ERS implica que el sistema implementado será el sistema
deseado. Se han seguido las recomendaciones del estándar IEEE 830 según la última versión del estándar citado. Las características deseables para una especificación de requisitos son:


\begin{enumerate}
	\item No ambigua.
	\item Completa
	\item  Verificable
	\item  Consistente
	\item  Clasificada
	\item  Modificable
	\item  Explorable
	\item  Utilizable durante las tareas de mantenimiento y uso
\end{enumerate}


\section{Objetivos generales}\label{objetivos-generales}
Los objetivos generales que se perseguían con el proyecto han sido:
\begin{itemize}
	\item Desarrollar una aplicación web que permita intercambiar turnos.
	\item Desarrollar una aplicación híbrida que adapte la anterior aplicación web a cualquier dispositivo.
\end{itemize}


\section{Catálogo de requisitos}\label{catalogo-requisitos}
A continuación, se enumeran los requisitos específicos:

En esta sección se especificarán los requisitos del sistema, diferenciando los requisitos funcionales, o sea el comportamiento del sistema, de los requisitos no funcionales, que describen características de funcionamiento.
\subsection{Requisitos funcionales}\label{r-funcionales}
\begin{description}
    \item[RF-1 Gestión de usuarios:] Los usuarios son una parte fundamental.
    \begin{itemize}
         \item \textbf{RF-1.1:} Registrar usuarios
         \item \textbf{RF-1.2:} Logear usuarios
         \item \textbf{RF-1.3:} Deslogearse
    \end{itemize}
   \item[RF-2. Gestión de calendario:] Funciona como un calendario normal.
    \begin{itemize}
         \item \textbf{RF-2.1:} Añadir turnos.
         \item \textbf{RF-2.2:} Eliminar turnos.
         \item \textbf{RF-2.3:} Editar turnos.
         \item \textbf{RF-2.4:} Visualizar turnos.
    \end{itemize}
 	   \item[RF-3. Gestión de cambios:] Parte de intercambio de turnos.
    \begin{itemize}
         \item \textbf{RF-3.1:} Enviar petición.
         \item \textbf{RF-3.2:} Aceptar petición.
         \item \textbf{RF-3.3:} Rechazar petición. 
         \item \textbf{RF-3.4:} Aceptar intercambio.
         \item \textbf{RF-3.5:} Rechazar intercambio.
         \item \textbf{RF-3.6:} Visualizar detalles.
           \begin{itemize}
             \item \textbf{RF-3.6.1:} Visualizar turnos.
             \item \textbf{RF-3.6.2:} Visualizar pendientes.
             \item \textbf{RF-3.6.3:} Visualizar aceptados.
             \item \textbf{RF-3.6.4:} Visualizar rechazados.
           \end{itemize}
         \item \textbf{RF-3.7:} Visualizar calendario.
    \end{itemize}
       
    
    
   \item[RF-4. Información:] Información de como usar la aplicación
\end{description}

\subsection{Requisitos no funcionales}\label{rnofuncionales}
\begin{description}
    \item [RNF-1 Seguridad] se persigue preservar la seguridad impidiendo acceder a la aplicación sin logearse. O también añadiendo un \emph{hash} para el almacenamiento de las contraseñas de los usuarios, preservando así la seguridad de los mismos.
    \item [RNF.2 Escalabilidad:] la aplicación debe estar preparada para actuar con un gran número de usuarios.
    \item [RNF.3 Soporte:]  la aplicación debe ser soportable por diferentes navegadores (Restricción: lamentablemente \emph{Angular 4} no es soportado actualmente por todos los navegadores del mercado)
    \item [RNF.4 Usabilidad:]  la aplicación debe ser fácil de usar.
\end{description}


\section{Especificación de requisitos}\label{requisitos}

En esta sección se explicará el diagrama de casos de uso y se desarrollará cada uno de los requisitos en función del esquema.

\subsection{Diagrama casos de uso}

%Imagen casos de uso
\imagen{casodeusoschema}{\footnotesize{Casos de uso. Fuente: Elaboración propia.}}

\subsection{Actores}
El actor es el usuario de la aplicación tenemos que tener en cuenta que para que se de un cambio de turno deben de existir dos actores sino sería imposible el intercambio.

\subsection{Casos de uso}

%%%%%%%%%%%%%%%%%%%%%%%%%%%
%CU-1: Gestión usuarios
\begin{table}[H]
\begin{tabular}{llr}  
\toprule
\begin{minipage}[b]{0.24\columnwidth}\raggedright\strut
\textbf{CU-01}\strut
\end{minipage} & \begin{minipage}[b]{0.72\columnwidth}\raggedright\strut
\textbf{Gestión usuarios}\strut
\end{minipage}\tabularnewline
\cmidrule(r){1-2}
\textbf{Versión}       & 1.0           \\
\textbf{Autor}       & Adrián  Aguado    \\
\textbf{Requisitos asociados}       & RF-1.1, RF-1.2, RF-1.3 \\ 
\textbf{Descripción} & Permite registrarse, logearse al usuario o salir.\\
\textbf{Precondición} & La página está cargada, \\
& base de datos funcionando.       \\
\textbf{Acciones posibles} & 1. El usuario se registra. \\
& 2. El usuario  se logea. \\
& 3. El usuario interactúa.\\
& 4. El usuario sale.        \\
\textbf{Postcondición} & Se añade el usuario \\
& a la base de datos.     \\
\textbf{Excepciones} &  \\
\textbf{Frecuencia} & Alta            \\
\textbf{Importancia} & Alta            \\
\textbf{Comentarios } &       \\
\bottomrule
\end{tabular}
\caption{CU-1 Gestión usuarios} 
\end{table}

%CU-02: Registro
\begin{table}[H]
\begin{tabular}{llr}  
\toprule
\begin{minipage}[b]{0.24\columnwidth}\raggedright\strut
\textbf{CU-02}\strut
\end{minipage} & \begin{minipage}[b]{0.72\columnwidth}\raggedright\strut
\textbf{Registro}\strut
\end{minipage}\tabularnewline
\cmidrule(r){1-2}
\textbf{Versión}       & 1.0           \\
\textbf{Autor}       & Adrián  Aguado    \\
\textbf{Requisitos asociados}       & RF-1.1 \\ 
\textbf{Descripción} & Permite registrarse.\\
\textbf{Precondición} & La página está cargada  \\
& en la pestaña adecuada.     \\
\textbf{Acciones} & 1. El usuario introduce los datos, \\
& a saber: \\
& A. Nombre \\
& B. Apellidos\\
& B. Nombre usuario\\
& C. E-mail\\
& D. Nombre empresa\\
& E. Tipo de turno (a elegir)\\
& F. Contraseña\\
\textbf{Postcondición} & Se añade el usuario \\
& a la base de datos.     \\
& Redirección a log-in.  \\
\textbf{Excepciones} &  Si no se introducen todos los campos  \\
& obligatorios se notifica. También si el usuario \\
& ya existe o si todo ha ido correctamente.    \\
\textbf{Frecuencia} & Alta            \\
\textbf{Importancia} & Alta            \\
\textbf{Comentarios } &       \\
\bottomrule
\end{tabular}
\caption{CU-02 Registro} 
\end{table}

%CU-03: Log-in
\begin{table}[H]
\begin{tabular}{llr}  
\toprule
\begin{minipage}[b]{0.24\columnwidth}\raggedright\strut
\textbf{CU-03}\strut
\end{minipage} & \begin{minipage}[b]{0.72\columnwidth}\raggedright\strut
\textbf{Log-in}\strut
\end{minipage}\tabularnewline
\cmidrule(r){1-2}
\textbf{Versión}       & 1.0           \\
\textbf{Autor}       & Adrián  Aguado    \\
\textbf{Requisitos asociados}       & RF-1.2 \\ 
\textbf{Descripción} & Permite logearse al usuario\\
\textbf{Precondición} & La página está cargada \\
& en la pestaña adecuada.     \\
& Base de datos funcionando.       \\
\textbf{Acciones} & 1. El usuario se \emph{logea}\\
& para ellos introduce:\\
& A. Correo electrónico \\
& B. Contraseña\\
& 2. Se cargan los turnos en el calendario.\\
\textbf{Postcondición} & El usuario entra en la  \\
& aplicación.    \\
\textbf{Excepciones} &  Si la contraseña no es correcta, aviso. \\
&  Si el correo no es correcto, aviso. \\
\textbf{Frecuencia} & Alta            \\
\textbf{Importancia} & Alta            \\
\textbf{Comentarios } &  Si fuera el primer \emph{log-in} el calendario   \\
& el calendario estará vacío.\\
\bottomrule
\end{tabular}
\caption{CU-03 Log-in} 
\end{table}

%CU-04: Log out
\begin{table}[H]
\begin{tabular}{llr}  
\toprule
\begin{minipage}[b]{0.24\columnwidth}\raggedright\strut
\textbf{CU-04}\strut
\end{minipage} & \begin{minipage}[b]{0.72\columnwidth}\raggedright\strut
\textbf{Log-out}\strut
\end{minipage}\tabularnewline
\cmidrule(r){1-2}
\textbf{Versión}       & 1.0           \\
\textbf{Autor}       & Adrián  Aguado    \\
\textbf{Requisitos asociados}       & RF-1.3 \\ 
\textbf{Descripción} & Permite salir de la aplicación.\\
\textbf{Precondición} & El usuario está dentro de la \\
& aplicación.       \\
\textbf{Acciones} & 1. Presionar el botón \emph{log out}.\\
   \\
\textbf{Postcondición} & Sale de la aplicación.  \\
\textbf{Excepciones} &     \\
\textbf{Frecuencia} & Alta            \\
\textbf{Importancia} & Alta            \\
\textbf{Comentarios } &      \\
\bottomrule
\end{tabular}
\caption{CU-04 Log out} 
\end{table}

Nota: El editar usuarios y eliminar usuarios no esta implementado de cara al usuario. Está implementado para realizar en una segunda fase  pero no se encuentra disponible en pantalla para realizarse directamente por parte del usuario.

%%%%%%%%%%%%%%%%%%%%%%%%%%%
%CU-05: Gestión Calendario
\begin{table}[H]
\begin{tabular}{llr}  
\toprule
\begin{minipage}[b]{0.24\columnwidth}\raggedright\strut
\textbf{CU-05}\strut
\end{minipage} & \begin{minipage}[b]{0.72\columnwidth}\raggedright\strut
\textbf{Gestión calendario}\strut
\end{minipage}\tabularnewline
\cmidrule(r){1-2}
\textbf{Versión}       & 1.0           \\
\textbf{Autor}       & Adrián  Aguado    \\
\textbf{Requisitos asociados}       & RF-2.1, RF-2.2, RF-2.3, RF-2.4  \\ 
\textbf{Descripción} & Gestión del calendario.\\
\textbf{Precondición} & El usuario está dentro de la \\
& aplicación.       \\
\textbf{Acciones} & 1. Añadir turnos. \\
& 2. Eliminar turnos. \\
& 3. Editar turnos. \\
& 4. Visualizar turnos. \\
\textbf{Postcondición} & Turno deseado. \\
\textbf{Excepciones} &     \\
\textbf{Frecuencia} & Media          \\
\textbf{Importancia} & Alta            \\
\textbf{Comentarios } &      \\
\bottomrule
\end{tabular}
\caption{CU-05 Gestión Calendario} 
\end{table}

%CU-06: Añadir turno
\begin{table}[H]
\begin{tabular}{llr}  
\toprule
\begin{minipage}[b]{0.24\columnwidth}\raggedright\strut
\textbf{CU-06}\strut
\end{minipage} & \begin{minipage}[b]{0.72\columnwidth}\raggedright\strut
\textbf{Añadir turno}\strut
\end{minipage}\tabularnewline
\cmidrule(r){1-2}
\textbf{Versión}       & 1.0           \\
\textbf{Autor}       & Adrián  Aguado    \\
\textbf{Requisitos asociados}       & RF-2.1 \\ 
\textbf{Descripción} & Añadir un turno determinado. \\
\textbf{Precondición} & El usuario está dentro de la \\
& aplicación.       \\
\textbf{Acciones} & 1. Presionar  botón \emph{nuevo turno}. \\
& 2. Seleccionar título. \\
& 3. Seleccionar tipo. \\
& 4 .Seleccionar fecha.\\
\textbf{Postcondiciones} & Turno deseado en calendario. \\
& Turno añadido en la base de \\
&  datos asociado al usuario \\
&  que lo ha añadido.  \\
&  Prioridad normal.  \\
\textbf{Excepciones} &  No es posible más de un turno al día.   \\
\textbf{Frecuencia} & Media          \\
\textbf{Importancia} & Alta            \\
\textbf{Comentarios } &     \\
\bottomrule
\end{tabular}
\caption{CU-06 Añadir turno} 
\end{table}




%CU-07: Eliminar turno
\begin{table}[H]
\begin{tabular}{llr}  
\toprule
\begin{minipage}[b]{0.24\columnwidth}\raggedright\strut
\textbf{CU-07}\strut
\end{minipage} & \begin{minipage}[b]{0.72\columnwidth}\raggedright\strut
\textbf{Eliminar turno}\strut
\end{minipage}\tabularnewline
\cmidrule(r){1-2}
\textbf{Versión}       & 1.0           \\
\textbf{Autor}       & Adrián  Aguado    \\
\textbf{Requisitos asociados}       & RF-2.3 \\ 
\textbf{Descripción} & Eliminar un turno.\\
\textbf{Precondición} & El usuario está dentro de la \\
& aplicación.       \\
\textbf{Acciones} & 1. Presionar botón \emph{eliminar turno} \\
& 2. Turno eliminado. \\
\textbf{Postcondiciones} & Turno eliminado del calendario.\\
& Turno eliminado de la base de datos. \\
\textbf{Excepciones} &     \\
\textbf{Frecuencia} & Media          \\
\textbf{Importancia} & Alta            \\
\textbf{Comentarios } &      \\
\bottomrule
\end{tabular}
\caption{CU-07 Eliminar turno} 
\end{table}

%CU-08: Editar turno
\begin{table}[H]
\begin{tabular}{llr}  
\toprule
\begin{minipage}[b]{0.24\columnwidth}\raggedright\strut
\textbf{CU-08}\strut
\end{minipage} & \begin{minipage}[b]{0.72\columnwidth}\raggedright\strut
\textbf{Editar turno}\strut
\end{minipage}\tabularnewline
\cmidrule(r){1-2}
\textbf{Versión}       & 1.0           \\
\textbf{Autor}       & Adrián  Aguado    \\
\textbf{Requisitos asociados}       & RF-2.3 \\ 
\textbf{Descripción} & Editar un turno.\\
\textbf{Precondición} & El usuario está dentro de la \\
& aplicación. \\
\textbf{Acciones} & 1.El usuario visualiza el turno  \\
& en la pestaña  \emph{normal}. \\
& 2. El usuario puede modificar :      \\
& A.Titulo \\
& B.Tipo \\
& C.Fecha\\
\textbf{Postcondición} & Turno deseado editado.\\
\textbf{Excepciones} &     \\
\textbf{Frecuencia} & Media          \\
\textbf{Importancia} & Alta            \\
\textbf{Comentarios } &    El usuario puede modificar  \\
&  un turno siempre que  \\
&  así lo desee.  \\
\bottomrule
\end{tabular}
\caption{CU-08 Editar turno} 
\end{table}

%CU-09: Visualizar turnos
\begin{table}[H]
\begin{tabular}{llr}  
\toprule
\begin{minipage}[b]{0.24\columnwidth}\raggedright\strut
\textbf{CU-09}\strut
\end{minipage} & \begin{minipage}[b]{0.72\columnwidth}\raggedright\strut
\textbf{ Visualizar turnos}\strut
\end{minipage}\tabularnewline
\cmidrule(r){1-2}
\textbf{Versión}       & 1.0           \\
\textbf{Autor}       & Adrián  Aguado    \\
\textbf{Requisitos asociados}       & RF-2.4 \\ 
\textbf{Descripción} & Ver de un simple vistazo \\
&todos los turnos. \\
\textbf{Precondición} & El usuario está dentro de la \\
& aplicación.       \\
\textbf{Acciones} & 1. Ver turnos. \\
\textbf{Postcondición} &   \\
\textbf{Excepciones} &     \\
\textbf{Frecuencia} & Media          \\
\textbf{Importancia} & Alta            \\
\textbf{Comentarios } &      \\
\bottomrule
\end{tabular}
\caption{CU-09 Visualizar turnos} 
\end{table}

%%%%%%%%%%%%%%%%%%%%%%%%%%%
%CU-10: Gestión cambios
\begin{table}[H]
\begin{tabular}{llr}  
\toprule
\begin{minipage}[b]{0.24\columnwidth}\raggedright\strut
\textbf{CU-10}\strut
\end{minipage} & \begin{minipage}[b]{0.72\columnwidth}\raggedright\strut
\textbf{Gestión cambios}\strut
\end{minipage}\tabularnewline
\cmidrule(r){1-2}
\textbf{Versión}       & 1.0           \\
\textbf{Autor}       & Adrián  Aguado    \\
\textbf{Requisitos asociados}       & RF-3.1, RF-3.2, RF-3.3, RF-3.4  \\ 
& RF-3.5, RF-3.6,RF-3.6.1, RF-3.6.2 \\ 
& RF-3.6.3, RF-3.6.4,  RF-3.7 \\ 
\textbf{Descripción} & Gestión de los cambios \\
& entre usuarios. \\
\textbf{Precondición}  & El usuario está dentro de la \\
& aplicación.      \\
& Hay más de un usuario. \\
\textbf{Acciones} & 1. Enviar petición de cambio. \\
& 2. Aceptar petición de cambio. \\
& 3. Rechazar petición de cambio. \\
& 4. Aceptar intercambio. \\
& 5. Rechazar intercambio.  \\
& 6. Visualizar detalles.  \\
& 7. Visualizar calendario. \\
\textbf{Postcondición} & Cambio turno. \\
\textbf{Excepciones} &   Para diversos casos  \\
 &   se muestran avisos.  \\
\textbf{Frecuencia} & Media          \\
\textbf{Importancia} & Alta            \\
\textbf{Comentarios } & Se controlan que la       \\
& petición se envíe una sola vez. \\
& Se muestra la lista de usuarios con \\
& turnos libres asociada a un \\
& día concreto \\
\bottomrule
\end{tabular}
\caption{CU-10 Gestión Cambios} 
\end{table}

%CU-11: Enviar petición
\begin{table}[H]
\begin{tabular}{llr}  
\toprule
\begin{minipage}[b]{0.24\columnwidth}\raggedright\strut
\textbf{CU-11}\strut
\end{minipage} & \begin{minipage}[b]{0.72\columnwidth}\raggedright\strut
\textbf{Envío de petición de cambio}\strut
\end{minipage}\tabularnewline
\cmidrule(r){1-2}
\textbf{Versión}       & 1.0           \\
\textbf{Autor}       & Adrián  Aguado    \\
\textbf{Requisitos asociados}       & RF-3.1 \\
\textbf{Descripción} & Gestión de los turnos \\
& entre usuarios. \\
\textbf{Precondición}  & El usuario está dentro de la \\
& aplicación.      \\
& Hay más de un usuario. \\
\textbf{Acciones} & 1. Enviar petición de cambio. \\
\textbf{Postcondición} & Ver si el usuario . \\
 & destinatario a aceptado o rechazado la petición. \\
\textbf{Excepciones} &   Aviso de envío de petición  \\
 &  tan solo una vez. \\
 &  Si no hay usuarios libres no es  \\
  &  posible enviar la petición  \\
\textbf{Frecuencia} & Media          \\
\textbf{Importancia} & Alta            \\
\textbf{Comentarios } & Se controlan que la       \\
& petición se envíe una sola vez. \\
& Se tienen en cuenta todos los usuarios      \\
& con turnos libre. \\
\bottomrule
\end{tabular}
\caption{CU-11 Enviar petición} 
\end{table}

%CU-12: Aceptar petición
\begin{table}[H]
\begin{tabular}{llr}  
\toprule
\begin{minipage}[b]{0.24\columnwidth}\raggedright\strut
\textbf{CU-12}\strut
\end{minipage} & \begin{minipage}[b]{0.72\columnwidth}\raggedright\strut
\textbf{Aceptar de petición de cambio}\strut
\end{minipage}\tabularnewline
\cmidrule(r){1-2}
\textbf{Versión}       & 1.0           \\
\textbf{Autor}       & Adrián  Aguado    \\
\textbf{Requisitos asociados}       & RF-3.1 \\
\textbf{Descripción} & Aceptación de la petición\\
\textbf{Precondición}  & El usuario está dentro de la \\
& aplicación.      \\
& Hay una petición por parte \\
& de otro usuario \\
\textbf{Acciones} & 1. Aceptar petición. \\
\textbf{Postcondición} & Cambio aceptado. \\
\textbf{Excepciones} &   Petición única  \\
\textbf{Frecuencia} & Media          \\
\textbf{Importancia} & Alta            \\
\textbf{Comentarios } & Se controlan que la       \\
& petición sea única. \\
\bottomrule
\end{tabular}
\caption{CU-12 Aceptar petición} 
\end{table}

%CU-13: Rechazar petición
\begin{table}[H]
\begin{tabular}{llr}  
\toprule
\begin{minipage}[b]{0.24\columnwidth}\raggedright\strut
\textbf{CU-13}\strut
\end{minipage} & \begin{minipage}[b]{0.72\columnwidth}\raggedright\strut
\textbf{Rechazar de petición de cambio}\strut
\end{minipage}\tabularnewline
\cmidrule(r){1-2}
\textbf{Versión}       & 1.0           \\
\textbf{Autor}       & Adrián  Aguado    \\
\textbf{Requisitos asociados}       & RF-3.2 \\
\textbf{Descripción} & Rechazo de la petición\\
\textbf{Precondición}  & El usuario está dentro de la \\
& aplicación.      \\
& Hay una petición por parte \\
& de otro usuario \\
\textbf{Acciones} & 1. Rechazar petición. \\
\textbf{Postcondición} & Fin de la lógica. \\
\textbf{Excepciones} &     \\
\textbf{Frecuencia} & Media          \\
\textbf{Importancia} & Alta            \\
\textbf{Comentarios } & Se controlan que la       \\
& petición sea única. \\
\bottomrule
\end{tabular}
\caption{CU-13 Rechazar petición} 
\end{table}

%CU-14: Aceptar intercambio
\begin{table}[H]
\begin{tabular}{llr}  
\toprule
\begin{minipage}[b]{0.24\columnwidth}\raggedright\strut
\textbf{CU-14}\strut
\end{minipage} & \begin{minipage}[b]{0.72\columnwidth}\raggedright\strut
\textbf{Aceptar intercambio}\strut
\end{minipage}\tabularnewline
\cmidrule(r){1-2}
\textbf{Versión}       & 1.0           \\
\textbf{Autor}       & Adrián  Aguado    \\
\textbf{Requisitos asociados}       & RF-3.3 \\
\textbf{Descripción} & Aceptación intercambio \\
\textbf{Precondición}  & El usuario está dentro de la \\
& aplicación.      \\
& Hay una petición por parte \\
& de otro usuario \\
\textbf{Acciones} & 1. Aceptar cambio. \\
\textbf{Postcondición} & Se produce el cambio. \\
& Los eventos se intercambian tanto \\
& en la base de datos como en el \\
& calendario de ambos usuarios \\
\textbf{Excepciones} &     \\
\textbf{Frecuencia} & Media          \\
\textbf{Importancia} & Alta            \\
\textbf{Comentarios } &   \\
\bottomrule
\end{tabular}
\caption{CU-14 Aceptar intercambio} 
\end{table}

%CU-15: Rechazar intercambio
\begin{table}[H]
\begin{tabular}{llr}  
\toprule
\begin{minipage}[b]{0.24\columnwidth}\raggedright\strut
\textbf{CU-15}\strut
\end{minipage} & \begin{minipage}[b]{0.72\columnwidth}\raggedright\strut
\textbf{Rechazar de petición de cambio}\strut
\end{minipage}\tabularnewline
\cmidrule(r){1-2}
\textbf{Versión}       & 1.0           \\
\textbf{Autor}       & Adrián  Aguado    \\
\textbf{Requisitos asociados}       & RF-3.4 \\
\textbf{Descripción} & Rechazo de la petición\\
\textbf{Precondición}  & El usuario está dentro de la \\
& aplicación.      \\
& Hay una petición por parte \\
& de otro usuario \\
\textbf{Acciones} & 1. Rechazar intercambio. \\
\textbf{Postcondición} & Fin de la lógica. \\
\textbf{Excepciones} &     \\
\textbf{Frecuencia} & Media          \\
\textbf{Importancia} & Alta            \\
\textbf{Comentarios } &  \\
\bottomrule
\end{tabular}
\caption{CU-15: Rechazar intercambio} 
\end{table}

%CU-16: Visualizar detalles 
\begin{table}[H]
\begin{tabular}{llr}  
\toprule
\begin{minipage}[b]{0.24\columnwidth}\raggedright\strut
\textbf{CU-16}\strut
\end{minipage} & \begin{minipage}[b]{0.72\columnwidth}\raggedright\strut
\textbf{Detalles peticiones}\strut
\end{minipage}\tabularnewline
\cmidrule(r){1-2}
\textbf{Versión}       & 1.0           \\
\textbf{Autor}       & Adrián  Aguado    \\
\textbf{Requisitos asociados}       & RF-3.6, RF-3.6.1  \\
& RF-3.6.2, RF-3.6.3, RF-3.6.4   \\
\textbf{Descripción} & Detalles de las peticiones\\
& mediante botones para una mejor \\
& comprensión por parte del usuario\\
\textbf{Precondición}  & El usuario está dentro de la \\
& aplicación.      \\
\textbf{Acciones} & 1. Visualizar normales. \\
& 2. Visualizar pendientes. \\
& 3. Visualizar aceptados. \\
& 4. Visualizar rechazados. \\
\textbf{Postcondición} & Ver numero de cada \\
& parámetro \\
\textbf{Excepciones} &     \\
\textbf{Frecuencia} & Media          \\
\textbf{Importancia} & Media           \\
\textbf{Comentarios } &  No es estrictamente necesaria \\
&  esta parte pero ayuda al usuario \\
\bottomrule
\end{tabular}
\caption{CU-16: Visualizar detalles} 
\end{table}

%CU-17: Visualizar normales
\begin{table}[H]
\begin{tabular}{llr}  
\toprule
\begin{minipage}[b]{0.24\columnwidth}\raggedright\strut
\textbf{CU-17}\strut
\end{minipage} & \begin{minipage}[b]{0.72\columnwidth}\raggedright\strut
\textbf{Detalles turnos}\strut
\end{minipage}\tabularnewline
\cmidrule(r){1-2}
\textbf{Versión}       & 1.0           \\
\textbf{Autor}       & Adrián  Aguado    \\
\textbf{Requisitos asociados}       & RF-3.6.1  \\
\textbf{Descripción} & Detalle de turnos \\
\textbf{Precondición}  & El usuario está dentro de la \\
& aplicación y ha realizado algún turno   \\
\textbf{Acciones} & 1. Visualizar normales. \\
\textbf{Postcondición} &  \\
\textbf{Excepciones} &     \\
\textbf{Frecuencia} & Media          \\
\textbf{Importancia} & Media           \\
\textbf{Comentarios } &   \\
\bottomrule
\end{tabular}
\caption{CU-17: Visualizar normales} 
\end{table}

%CU-18: Visualizar pendientes
\begin{table}[H]
\begin{tabular}{llr}  
\toprule
\begin{minipage}[b]{0.24\columnwidth}\raggedright\strut
\textbf{CU-18}\strut
\end{minipage} & \begin{minipage}[b]{0.72\columnwidth}\raggedright\strut
\textbf{Detalles peticiones pendientes}\strut
\end{minipage}\tabularnewline
\cmidrule(r){1-2}
\textbf{Versión}       & 1.0           \\
\textbf{Autor}       & Adrián  Aguado    \\
\textbf{Requisitos asociados}       & RF-3.6.2  \\
\textbf{Descripción} & Detalle de peticiones \\
& pendientes \\
\textbf{Precondición}  & El usuario está dentro de la \\
& aplicación y ha solicitado algún   \\
& cambio de turno (petición pendiente)   \\
\textbf{Acciones} & 1. Visualizar peticiones pendientes. \\
\textbf{Postcondición} &  \\
\textbf{Excepciones} &     \\
\textbf{Frecuencia} & Media          \\
\textbf{Importancia} & Media           \\
\textbf{Comentarios } &   \\
\bottomrule
\end{tabular}
\caption{CU-18 Visualizar pendientes} 
\end{table}

%CU-19: Visualizar aceptadas
\begin{table}[H]
\begin{tabular}{llr}  
\toprule
\begin{minipage}[b]{0.24\columnwidth}\raggedright\strut
\textbf{CU-19}\strut
\end{minipage} & \begin{minipage}[b]{0.72\columnwidth}\raggedright\strut
\textbf{Visualizar aceptadas}\strut
\end{minipage}\tabularnewline
\cmidrule(r){1-2}
\textbf{Versión}       & 1.0           \\
\textbf{Autor}       & Adrián  Aguado    \\
\textbf{Requisitos asociados}       & RF-3.6.3  \\
\textbf{Descripción} & Detalle de peticiones \\
& aceptadas\\
\textbf{Precondición}  & El usuario está dentro de la \\
& aplicación y ha realizado alguna  \\
& petición de cambio.  \\
\textbf{Acciones} & 1. Visualizar peticiones\\
& aceptadas . \\
\textbf{Postcondición} &  Confirmar cambio \\
\textbf{Excepciones} &  Si la petición se ha rechazado   \\
&  no se visualizará aquí.   \\
\textbf{Frecuencia} & Media          \\
\textbf{Importancia} & Media           \\
\textbf{Comentarios } &   \\
\bottomrule
\end{tabular}
\caption{CU-19: Visualizar aceptadas} 
\end{table}

%CU-20: Visualizar rechazadas
\begin{table}[H]
\begin{tabular}{llr}  
\toprule
\begin{minipage}[b]{0.24\columnwidth}\raggedright\strut
\textbf{CU-20}\strut
\end{minipage} & \begin{minipage}[b]{0.72\columnwidth}\raggedright\strut
\textbf{Detalles peticiones}\strut
\end{minipage}\tabularnewline
\cmidrule(r){1-2}
\textbf{Versión}       & 1.0           \\
\textbf{Autor}       & Adrián  Aguado    \\
\textbf{Requisitos asociados}       & RF-3.6.4  \\
\textbf{Descripción} & Detalle de peticiones \\
& rechazadas \\
\textbf{Precondición}  & El usuario está dentro de la \\
& aplicación y ha realizado alguna  \\
& petición de cambio.  \\
\textbf{Acciones} & 1. Visualizar peticiones\\
& rechazadas . \\
\textbf{Postcondición} &  Rechazar cambio \\
\textbf{Excepciones} &  Si la petición se ha aceptado   \\
&  no se visualizará aquí.   \\
\textbf{Frecuencia} & Media          \\
\textbf{Importancia} & Media           \\
\textbf{Comentarios } &   \\
\bottomrule
\end{tabular}
\caption{CU-20 Visualizar rechazadas} 
\end{table}

%CU-21: Visualizar Calendario
\begin{table}[H]
\begin{tabular}{llr}  
\toprule
\begin{minipage}[b]{0.24\columnwidth}\raggedright\strut
\textbf{CU-21}\strut
\end{minipage} & \begin{minipage}[b]{0.72\columnwidth}\raggedright\strut
\textbf{Visualizar calendario}\strut
\end{minipage}\tabularnewline
\cmidrule(r){1-2}
\textbf{Versión}       & 1.0           \\
\textbf{Autor}       & Adrián  Aguado    \\
\textbf{Requisitos asociados}       & RF-3.6.4  \\
\textbf{Descripción} & Visualizar cambios en \\
& el  calendario \\
\textbf{Precondición}  & El usuario está dentro de la \\
& aplicación. \\
\textbf{Acciones} & 1. Visualizar turnos \\
& 2. Click encima de turnos para ver \\
&  duración  \\
&  3, Color asociado a cada tipo de turno  \\
\textbf{Postcondición} & \\
\textbf{Frecuencia} & Media          \\
\textbf{Importancia} & Media           \\
\textbf{Comentarios } &  Todos los cambios de turno  \\
 &  se deben de ver reflejados en el calendario  \\
  &  del usuario en caso de ser aceptados por  \\
    &  ambas partes \\
\bottomrule
\end{tabular}
\caption{CU-21 Visualizar calendario} 
\end{table}





