\capitulo{5}{Aspectos relevantes del desarrollo del proyecto}

Este apartado pretende recoger los aspectos más interesantes del desarrollo del proyecto: Decisiones que se han tomado, desarrollo, progreso general y problemas que surgieron durante toda la realización del proyecto.


\section{Inicio del proyecto}\label{inicio-proyecto}
Cuando llego la hora de buscar proyecto, entre las  diferentes opciones disponibles estaba \emph{SWAPP}, una herramienta realizada años atrás en la universidad por un alumno que desarrolló una aplicación nativa en Android sobre el mismo tema.

 Desde el primer momento me llamo la atención el proyecto por que resultaba, más allá de ser un proyecto fin de grado, un aplicación que en un futuro podía tener un futuro real. Además se trataba de tecnologías web, no se especificaban cual tras las cuales, un vez desarrollada la parte web, sería posible convertirlas en una aplicación híbrida. Me pareció una idea fascinante, sobre todo por que desconocía plataformas como \emph{Cordova} o \emph{Phohegapp}. Está claro que el principal peso del proyecto se lo lleva la aplicación web, dado que es la parte fundamental y sobre la que se va a sustentar todo lo demás. Por lo que resultaba fundamental escoger un buen framework web, estudiarlo convenientemente y comenzar a realizar un trabajo sólido a partir de ese momemento.
 
  En consecuencia surge el primero de los debates, que tipo de framework usar. La multitud de opciones que existe en el mercado es realmente es extensa, una de las opciones a tener en cuenta para la elección de tu aplicación es comprobar que el framework sigue recibiendo soporte actualmente (Por ejemplo github tiene "escaparates" donde agrupa los frameworks por tipos, algunos repositorios son \hyperlink{ https://github.com/showcases/web-application-frameworks/}{públicos}  y es posible saber la frecuencia con la que se actualizan ). Además una de las mejores cosas del software libre es que puedes interactuar con los propios creadores  o desarrolladores de una tecnologías, abrir \emph{issues} en github y en numerosas ocasiones recibes respuesta. 
  
  
  \section{Formación}\label{formacion}
Desde el primer momento se priorizó la formación por delante de todo ya que el proyecto requiere conocimientos de dos tipos: por una parte conocimientos web en los que el alumno debía especializarse ya que en la Universidad de Burgos no se ve como tal el desarrollo web en ninguna asignatura, por otro lado conocimientos que eran totalmente desconocidos para el alumno como manejo de la parte del servidor, bases de datos NoQL o desarrollo de aplicaciones híbridas. 

Gracias a las nuevas tecnologías tan solo es necesario un ordenador y una buena conexión a internet para tener acceso a conocimiento infinito. Desde hace algunos años esta en auge la formación online y las plataformas web que ofrecen cursos de diversos tipos. Existen numerosísimas plataformas de este estilo, una de las más conocidas es  \hyperlink{www.coursera.org}{Coursera}, que reúne cursos online masivos, abiertos y gratuitos (MOOCS). El curso fue recomendador por mi tutor y sin duda no se equivocaba, desde hace algún tiempo la visibilidad de la parte gratuita se ha reducido pudiendo solo acceder a partes limitadas de los contenidos. Es posible realizar el curso completo pero no obtener feedback de otros usuarios ni tampoco subir a la plataforma los avances de código.

Por lo que en un primer momento se decidió realizar la siguiente especialización compuesta de seis cursos diferentes:

\begin{description}
	\item[\emph{HTML, CSS y JavaScript}] Introducción al desarrollo web con las principales tecnologías. Qué es DOM y CSS.
	\item[ \emph{Front-End Web UI Frameworks and Tools}] Conocimiento de lo que es un framework web: Boostrap. También introducción a los preprocesadores Less, Sass.
	\item[\emph{Front-End JavaScript Frameworks: AngularJS} ]  Introducción y desarrollo con AngularJS.
	\item[ \emph{Multiplatform Mobile App Development with Web Technologies}] Desarrollo UI/UX, el framework Ionic como complemento a AngularJS para  crear aplicaciones hibridas con ayuda de Cordova..
	\item[ \emph{Server-side Development with NodeJS} ] Lado del servidor con Node JS.
	\item[ \emph{Full Stack Web Development Specialization Capstone}] Tabla con detalles de la especialización.
\end{description}


Los cursos pertenecer al curso \hyperlink{https://www.coursera.org/specializations/full-stack}{\emph{full-stack web develompent}} y son bastante completos en lo que a materia se refiere. Si es necesario saber algo de inglés por que sino se hacen bastante pesados.

Después además realicé algún que otro curso en español como son: 

 \section{Frameworks cliente}\label{cliente}
 
 \emph{AngularJs vs Angular 4} está fue una de las preguntas con las que más tiempo perdí al inicio del desarrollo y después de la fase de formación
 
  \section{Tipo de base de datos}\label{base de datos}
  
  Otro de los dilemsa sin duda fue la elección de la base de datos
 
 