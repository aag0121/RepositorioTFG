\capitulo{2}{Objetivos del proyecto}

En este apartado se van a detallar los diferentes objetivos que se buscaban, a diferentes niveles, con la realización del proyecto.

\section{Objetivos generales}\label{objetivos-generales}
Comenzaremos con los objetivos generales del proyecto. 
\begin{itemize}
\tightlist
\item
  Desarrollar una  \emph{aplicación web} que permita la gestión de turnos de una manera sencilla. 
\item
  Desarrollar una aplicación móvil híbrida asociada a la anterior que permita acceder desde cualquier plataforma a esa aplicación para dar un servicio multiplataforma. 
\item
  Permitir tener una herramienta que resuelva un problema real.
\end{itemize}

\section{Objetivos técnicos}\label{objetivos-tecnicos}
A continuación explicaré los principales objetivos técnicos que se pretendían. 
\begin{itemize}
\tightlist
\item
  Toma de contacto con el mundo web por parte del alumno.
 \item
  Aprendizaje de la implementación de una aplicación web con \emph{frameworks} web/móvil, como puede ser el caso de Angular o Ionic. 
\item
  Breve toma de contacto con el mundo móvil (\emph{PhoneGapp, Cordova}).
\item
 Aplicar la metología Scrum.
\end{itemize}

\section{Objetivos personales}\label{objetivos-personales}
Los objetivos personales que he perseguido durante todo el desarrollo han sido los siguientes:
\begin{itemize}
\tightlist
\item
 El diseño e implementación de una aplicación web, realizada con \emph{Node.js}y \emph{Express} para la parte del
servidor, y \emph{Angular} y \emph{Bootstrap} para la parte del cliente.  En definitiva aprender las diferencias fundamentales entre \emph{back} y \emph{front}.
\item
  Conocer y manejar bases de datos NoSQL, como \emph{MongoDB}.
\item
  Realizar un algoritmo desde cero para aplicarlo a un problema real.
\item
  Averiguar y aprender el mayor número de cosas sobre las tecnologías web del mercado actual. 
\item
  Aprender a realizar un proyecto de gran envergadura que sea aplicable a un problema real, y sobre todo, a gestionarlo.
\end{itemize}

\section{Objetivos alcanzados}\label{objetivos-alcanzados}
Los objetivos que finalmente se han alcanzado han sido: 
\begin{itemize}
\tightlist
\item
  Algoritmo funcionando con éxito.
\item
  Aplicación web. 
\item
  o3.
\end{itemize}
