\capitulo{2}{Objetivos del proyecto}

En este apartado se van a detallar los diferentes objetivos que se buscaban, a diferentes niveles, con la realización del proyecto.

\section{Objetivos generales}\label{objetivos-generales}
Comenzaremos con los objetivos generales del proyecto. 
\begin{itemize}
\tightlist
\item
  Desarrollar una  \emph{aplicación web} que permita la gestión de turnos de una manera sencilla e intuitiva. 
\item
  Desarrollar una aplicación móvil híbrida que permita acceder desde cualquier plataforma a esa aplicación para dar un servicio multiplataforma. 
\item
  Permitir tener una herramienta que resuelva un problema real.
\end{itemize}

\section{Objetivos técnicos}\label{objetivos-tecnicos}
A continuación explicaré los principales objetivos técnicos que se pretendían. 
\begin{itemize}
\tightlist
\item
  Toma de contacto con el mundo web por parte del alumno, a su juicio el dónde . 
 \item
  Aprendizaje de la implementación de una aplicación web con  \emph{frameworks} web, como puede ser el caso de Angular o Ionic. 
\item
  Breve toma de contacto con el mundo móvil (\emph{PhoneGapp, Cordova}).
\item
 Aplicar la metología Scrum.
 \item
 Utilizar ZenHub como Herramienta de gestión de proyectos
\end{itemize}

\section{Objetivos personales}\label{objetivos-personales}
Los objetivos personales que he perseguido durante todo el desarrollo han sido los siguientes:
\begin{itemize}
\tightlist
\item
  Mejorar mi inquietud y curiosidad sobre el mundo web. 
\item
  Averiguar y aprender el mayor número de cosas sobre las tecnologías web crecientes. 
\item
  Aprender a realizar un proyecto de gran envergadura, y sobre todo, a gestionarlo.
\end{itemize}

\section{Objetivos alcanzados}\label{objetivos-alcanzados}
Los objetivos que finalmente se han alcanzado han sido: 
\begin{itemize}
\tightlist
\item
  Aplicación web funcionando. 
\item
  Desarrollar una \emph{aplicación web} . 
\item
  Hola hola.
\end{itemize}

\section{Objetivos futuros}\label{objetivos-futuros}
Los objetivos que se pretenden seguir en el futuro: 
\begin{itemize}
\tightlist
\item
  Lograr una mejora sustanciar de la usabilidad de la aplicación en su versión móvil. 
\item
  Realizar encuesta para mejorar la aplicación base en función de lo que lo usuarios demanden . 
\item
  Implementar un chat que permita poner en contacto directamente a los usuarios.
\item
  Posicionar la herramienta dentro del mercado.
\end{itemize}